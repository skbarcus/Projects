\documentclass[oneside]{book}   %Oneside makes the document one sided. Without this chapters always start on odd numbered pages.
\usepackage[utf8]{inputenc}
\usepackage[english]{babel}
 
\usepackage[bookmarksopen]{hyperref}   %Bookmarks adds index on the sidebar in the PDF viewer.
\usepackage{indentfirst}
\usepackage{graphicx}
\usepackage{longtable}
\usepackage{multirow,bigstrut}
\usepackage{caption}
\usepackage{cleveref} %Load this package last.
 
\hypersetup{
    colorlinks=true,
    linkcolor=blue,
    filecolor=magenta,      
    urlcolor=cyan,
    pdftitle={Sharelatex Example},
    bookmarks=true,
    hyperindex=true,
    pdfpagemode=UseOutlines,
    pdfstartpage=1,
}

%\setlength{\parindent}{10ex}
\usepackage{xspace}    %Guesses if a space is needed after the custom command.
\usepackage{amsmath}   %Use \text in math mode.
\usepackage{makecell}  %Allows for the making of cells in tables.
\usepackage{caption}   %Lets me call a figure a table.

\newcommand{\hcal}{HCAL-J\xspace}
\newcommand{\jlab}{Jefferson Lab\xspace}
\newcommand{\q}{$Q^2$\xspace}

\begin{document}
 
\frontmatter

\begin{titlepage} % Suppresses displaying the page number on the title page and the subsequent page counts as page 1
	\newcommand{\HRule}{\rule{\linewidth}{0.5mm}} % Defines a new command for horizontal lines, change thickness here
	
	\center % Centre everything on the page
	

	%------------------------------------------------
	%	Title
	%------------------------------------------------
	
	\HRule\\[0.4cm]
	
	{\huge\bfseries Regarding Lands}\\[0.4cm] % Title of your document
	\textsc{\Large A Treatise on Lands in Limited}
	\HRule\\[1.5cm]
	
	%------------------------------------------------
	%	Headings
	%------------------------------------------------
	
	\textsc{\LARGE Tools for answering questions like:}\\[1.cm] % Main heading such as the name of your university/college
	
	\textsc{\Large How many lands should I play?}\\[0.5cm] % Minor heading such as course title
	
	\textsc{\Large Should I keep this hand?}\\[1.5cm] % Major heading such as course name
	%------------------------------------------------
	%	Author(s)
	%------------------------------------------------
	
	\begin{minipage}{0.5\textwidth}
		\begin{center}
			\large
			%\textit{Contact Person:}
			Scott \textsc{Barcus}\footnote{email: scottkbarcus@gmail.com} \newline
		\end{center}
	\end{minipage}
	
	
	% If you don't want a supervisor, uncomment the two lines below and comment the code above
	%{\large\textit{Author}}\\
	%John \textsc{Smith} % Your name
	
	%------------------------------------------------
	%	Date
	%------------------------------------------------
	
	\vfill\vfill\vfill % Position the date 3/4 down the remaining page
	
	{\large\today} % Date, change the \today to a set date if you want to be precise
	
	%------------------------------------------------
	%	Logo
	%------------------------------------------------
	
	%\vfill\vfill
	%\includegraphics[width=0.2\textwidth]{placeholder.jpg}\\[1cm] % Include a department/university logo - this will require the graphicx package
	 
	%----------------------------------------------------------------------------------------
	
	\vfill % Push the date up 1/4 of the remaining page
	
\end{titlepage}

\tableofcontents
 
\mainmatter
 
\chapter{Introduction: Limited Land Counts}
\label{intro}

Everyone who has ever shuffled up a 40 card limited deck has been confronted with the same question. How many lands should I run? And every time they draw from that deck, another question arises. Should I keep this hand? As with most things in Magic the Gathering, the answer to these questions is generally, ``It depends."

Throughout Magic's history it has generally come to be accepted that the default number of lands in a 40 card limited deck is 17. But where did the number 17 come from? Why is that the `optimal' number of lands. The answer of course, is that it is not always the optimal number of lands to run. Slower, more powerful limited formats, may incentivize playing more than 17 lands. Whereas, fast formats, where board early presence dominates, may incentivize playing fewer than 17 lands. Similarly, a player with a classical `control' deck may wish to run more than 17 lands to be certain to hit all of their land drops for their powerful, but generally more expensive, spells. A player with a lower curve `aggro' deck, may instead want to run fewer than 17 lands. 

Once you have decided how many lands your limited deck should run you draw your opening seven and see just a single land. Conventional Magic wisdom is that you generally mulligan one-landers, but to be the best limited player one can, one must be able to tell when conventional heuristics are wrong. Maybe you have two strong two drops in your agrro deck's one land opening hand and are on the play. What if you are on the draw? Maybe your control deck must hit its first four or five land drops on curve to be competitive. Can you keep a two land opener on the draw? 

To answer these questions you need to know your deck and how it is trying to win. You need to know how your game plan influences the number of lands you want to draw and when you want to draw them. No amount of tables and graphs can replace the hard-won experience earned from slinging spells against clever opponents. This document contains the statistics governing the answers to many of the above questions. This document cannot, however, answer these questions for you. Instead this document aims to give you the foundational statistical tools you need to make informed decisions regarding lands in limited.

\section{The MTGA Best of One Shuffler}
\label{sec:computers_overview}

The number of people playing MtG digitally has greatly increased with the release of Magic the Gathering Arena. With the advent of MTGA came a new emphasis on best-of-one (Bo1) games. For these MTGA Bo1 games, Wizard's introduced the Bo1 shuffler to reduce the number of non-interactive games. Unlike in traditional Bo3 games, MTGA Bo1 games draws two opening seven card hands from your deck when a game begins. The Bo1 shuffler then calculates which of these two hands to keep based on which of the hands has a land ratio closest to that of your deck overall. MTGA then presents you with the hand with the closer land ratio to that of your deck as your opening hand. (Note: You never see the other hand.)

For example, if we have a 40 card limited deck with 17 lands in it our deck is 42.5\% (17/40) lands. Now assume the Bo1 shuffler draws one hand with two lands and five nonlands and one hand with three lands and four nonlands. The first hand is 28.6\% (2/7) lands and the second hand is 42.9\% (3/7) lands. In this case the Bo1 shuffler will select the second hand with three lands in it because 42.9\% lands is closer to the deck's land ratio of 42.5\% than the two land hand's 28.6\%. 

The theory behind the Bo1 shuffler is that you will have more opening hands closer to your deck's land percentage. This greatly reduces the variance in opening hands, which leads to more keepable opening hands, and hopefully more fun interactive games of Magic. This has led to the belief among many players that it is better to default to playing 16 lands in limited when playing Bo1 games. 

The argument goes that running 17 lands in a 40 card deck leads to drawing more lands over the whole game than desired (flooding). The traditional reason to run 17 lands in limited is for color considerations and to make certain that one hits one's important land drops on curve (i.e. having three lands on turn three). But if the Bo1 shuffler causes us to draw more three land opening hands, even with only 16 lands in a deck, then the risk of not hitting land drops on curve is greatly reduced, and one can play fewer lands to prevent late game flood. This document will present statistics for Bo1 opening hands as well as traditional opening hands so that readers can make an informed decision when deciding how many lands to run.

\chapter{Results and Analysis}
\label{results}

The values given in the following tables and plots are analytic\footnote{Tables and plots were produced with Python, SciPi, NumPy, Matplotlib, and Seaborn.}, however the opening hand alnd percentages were cross checked against a Monte Carlo simulation as a sanity check. The two methods were found to be in good agreement. When using these results remember, every decision you make must be made with the composition of your deck in mind. Only when you understand the game plan of your deck can statistics help you to make an informed decision.

\section{Opening Hand Land Statistics}
\label{opener}

The most fundamental statistic governing the number of lands to put in a deck is the average number of lands you can expect to draw in an opening hand. Table \ref{fig:opening_hand_percentages} shows the percentage of seven card hands that will have a given number of lands in them for 40 card limited deck. The values are given for decks with between 14 and 20 lands, as well as for traditional draws and Bo1 draws.  

	\begin{figure}[!ht]
	\centering
	\centerline{\includegraphics[width=1.5\linewidth]{/home/skbarcus/Projects/MtG/MTGA_Opening_Hands/Pictures/Analytic_Opening_Land_Percentages_Clean.png}}
	\captionof{table}{
	{\bf{Average Opening Hand Land Percentages.}} This table gives the percentage chance of drawing between zero and seven lands in an opening hand of a 40 card limited deck. Odds are given for 14 to 20 lands in a deck, as well as for a traditional Magic opening hand draw and an MTGA Bo1 draw.}
	\label{fig:opening_hand_percentages}
	\end{figure}	
	
Here we see right away why 17 lands is considered the default number of lands for a 40 card limited deck to run when using the traditional opening hand draw. Putting 17 lands in your limited deck gives you the highest chance (32.30\%) of having three lands in your opening hand. Having three lands in your opener guarantees you hit the extremely important land drops on turns one to three. However, look at what happens when we use an MTGA Bo1 opening hand. Suddenly our odds of having three lands in our opener with 17 lands in the deck is 54.16\%. That's an enormous increase! What's going on here?

To understand why the opening hand land odds changed so much let's plot a histogram of the chances of drawing between zero and seven lands in your opening hand for a traditional draw and a MTGA Bo1 draw as in Figure \ref{fig:traditional_vs_bo1_17}. The effect of the MTGA Bo1 draw is that it `tightened up' the distribution centered around drawing three lands in your opener. Because the Bo1 shuffler draws two hands and keeps the one with the land ratio closest to that of your deck you will draw more three land hands. This decreases the likelihood of drawing a land ratio in your opener that is very different than the land ratio of your deck. You were already less likely to draw a six land opener than a three land opener with a traditional draw, but now if you draw a six land opener (85.8\% lands) as one of your Bo1 opening hands then the second hand will be chosen instead unless it has zero (0\% lands), six (85.8\% lands) or seven lands (100\% lands) in it. This is because these are the only land ratios further away from that of your deck (42.5\% lands). The overall effect is to greatly decrease the number of opening hands that have land ratios much different than your overall deck. In fact, drawing a seven land opener for a 40 card limited deck with 17 lands in MTGA Bo1 games only happens one in 918,984 times!

 	\begin{figure}[!ht]
	\centering
	\centerline{\includegraphics[width=1.\linewidth]{/home/skbarcus/Projects/MtG/MTGA_Opening_Hands/Pictures/17_Lands_Normal_vs_Bo1.png}}
	\caption{
	{\bf{Opening Hand Land Percentages Traditional Opening Draw vs. MTGA Bo1 Draw.}} Odds are given for a 40 card limited deck with 17 Lands.}
	\label{fig:traditional_vs_bo1_17}
	\end{figure}	
	
Let's take this a step further and see how the odds of drawing between zero and seven lands in an opening hand vary for 40 card limited decks with 14 to 20 lands in them. First we'll plot\footnote{Plots were made using Seaborn's kdeplot function and adjusting the bandwidth parameter until the distributions were clear.} the distributions for only traditional opening hand draws as in Figure \ref{fig:traditional_distributions}. Notice how the most frequent number of lands in an opening hand smoothly increases from 14 land decks up to 20 land decks. The bumps on the plot are due to the discrete nature of drawing cards. You can only draw an integer number of lands in an opening hand (i.e. you cannot draw 2.5 lands in a hand). As we saw in the table the distribution with the highest peak at three lands in an opening hand is a deck with 17 lands (red curve).

 	\begin{figure}[!ht]
	\centering
	\centerline{\includegraphics[width=1.\linewidth]{/home/skbarcus/Projects/MtG/MTGA_Opening_Hands/Pictures/Distribution_of_Lands_Normal_Opener_kdeplot_14-20.png}}
	\caption{
	{\bf{Opening Hand Land Percentages Using a Traditional Draw.}} Odds are given for a 40 card limited deck with between 14 and 20 Lands.}
	\label{fig:traditional_distributions}
	\end{figure}	

This all seems rather intuitive so far. If we start with only 14 lands in a deck and begin adding lands the average number of lands in our starting hand will increase in a relatively smooth fashion. Now let's see what these distributions look like for the MTGA Bo1 opening hands which are shown in Figure \ref{fig:bo1_distributions}? 

 	\begin{figure}[!ht]
	\centering
	\centerline{\includegraphics[width=1.\linewidth]{/home/skbarcus/Projects/MtG/MTGA_Opening_Hands/Pictures/Distribution_of_Lands_MTGA_Bo1_Opener_kdeplot_14-20.png}}
	\caption{
	{\bf{Opening Hand Land Percentages Using an MTGA Bo1 Draw.}} Odds are given for a 40 card limited deck with between 14 and 20 Lands.}
	\label{fig:bo1_distributions}
	\end{figure}	
	
Figure \ref{fig:bo1_distributions} has a strikingly different distribution than Figure \ref{fig:traditional_distributions}. Firstly, the distributions for each opening hand are much tighter than in the case of a traditional opening hand draw. This is a result of the decreased width of the distributions based on the decreased likelihood of drawing an opening hand with a land ratio that differs significantly from the land ratio of the overall deck (\ref{fig:traditional_vs_bo1_17}). 

Secondly, we no longer see a smooth transition of the land distribution of the opening hand from 14 lands in the deck up to 20 lands. What we see is three clearly separately grouped distributions: one for 14 lands, one for 15-19 lands, and one for 20 lands. Why do these distributions not smoothly flow into one another as the did for traditional opening hand draws? The answer lies in what number of lands in the opening hand is closest to the land ratio of the deck overall. 

A 14 land 40 card limited deck has a land percentage of 35\% (14/40). So the number of lands in an opening hand that is closest to a deck that is 35\% land is 2.45 lands (0.35$\times$7). 2.45 lands rounds down to two lands so the MTGA Bo1 draw will pick two land openers as the best match for a 14 land deck. The MTGA Bo1 draw would view the ideal number of lands in an opener for a 15 land deck as 2.625 lands (0.375$\times$7), for a 16 land deck as 2.8 lands (0.4$\times$7), for a 17 land deck as 2.975 (0.425$\times$7), for an 18 land deck as 3.15 lands (0.45$\times$7), and for a 19 land deck as 3.325 lands (0.475$\times$7). For 15-19 lands in a 40 card deck the MTGA Bo1 draw will round to three lands in the opening hand as ideal. Finally, for 20 lands the deck the ideal lands in the opener to the MTGA Bo1 draw is 3.5 lands (0.5$\times$7) which rounds up to four lands in the opener. This is what creates the three distinct groupings in Figure \ref{fig:bo1_distributions}. The overall effect of the MTGA Bo1 shuffler is to greatly increase the probability of drawing a number of lands in your opener closest to the land ratio of your deck. 

Most decks have an acceptable range for the number of lands they draw in their opener. For example, a many limited decks could easily consider keeping between two and four lands in its opener. Once you know your deck and its game plan you will have an idea of how many land would be ideal in your opener as well as how many lands you would be fairly happy to accept. Table \ref{fig:combined} gives the probabilities of drawing a given range of lands in your opener. If you know that your deck would generally be able to consider keeping a hand with say 2-4 lands in it opener, this table will give you the probabilities of drawing an opener with a number of lands in the given `acceptable' range.

 	\begin{figure}[!ht]
	\centering
	\centerline{\includegraphics[width=0.7\linewidth]{/home/skbarcus/Projects/MtG/MTGA_Opening_Hands/Pictures/Analytic_Combined_Opening_Land_Percentages_Clean.png}}
	\captionof{table}{
	{\bf{Probabilities of Drawing a Given Range of Lands in and Opening Hand.}} This table gives the probabilities of drawing . `\% 2-3 Lands' means the percentage chance of drawing either two or three lands in your opener. `14 Lands' means a 40 card limited deck with 14 lands using a traditional opening draw, and `14 Lands Bo1' is the same except for using the MTGA Bo1 opening draw.}
	\label{fig:combined}
	\end{figure}	

\section{Making Land Drops on Curve}
\label{curve}

Now we know the probability of drawing zero to seven lands in our opener for traditional draws and MTGA Bo1 draws. We also know that the MTGA Bo1 draws significantly alter the probabilities of having a given number of lands in an opener. This helps us to decide how many lands to run in a 40 card limited deck. But just maximizing your chances of drawing say three lands in your opener is by no means the whole story. Why do we want three lands in our opener anyway? The reason is that we want to hit our critical land drops on curve (i.e. two lands on turn two, three lands on turn three, etc.). Different decks will value hitting their land drops on curve differently. For example, an aggro deck might `need' two lands on turn two but could miss three lands on turn three and still function. Whereas, a control deck may `need' three lands on turn three or four lands on turn four to be functional. 

We can put the perfect number of lands in our deck, but no matter how perfect its land distribution you are often not going to draw the exact number of lands you desire in your opener. Once you draw a two land opener you then have to decide whether or not to keep that hand. One of the biggest factors in the decision to keep or mulligan a hand is how likely you are to make your land drops on curve. There are of course many other considerations such as the power level of the cards in your opener, their converted mana cost (CMC), and how well the cards fit your deck's game plan to name a few, but we will limit this analysis to focusing on lands and again remind the reader that everything in magic is contextual.

One further consideration when calculating the probabilities of a deck hitting its land drops on curve is whether that deck's player is on the play or draw. When a player goes first in a standard limited game that player does not draw a card on their first turn, whereas the player who goes second does. This means that if you are on the play you have one fewer draw to find a land to hit you land drops on curve. 

With that said, the next portion of this document gives tables of the probabilities of making your land drops on curve based on the number of lands in your deck, the number of lands in your opener, and whether or not you are on the play or draw. Table \ref{fig:14_curve} shows the probabilities for a 14 land limited deck, Table \ref{fig:15_curve} shows the probabilities for a 15 land limited deck, Table \ref{fig:16_curve} shows the probabilities for a 16 land limited deck, Table \ref{fig:17_curve} shows the probabilities for a 17 land limited deck, Table \ref{fig:18_curve} shows the probabilities for a 18 land limited deck, Table \ref{fig:19_curve} shows the probabilities for a 19 land limited deck, and Table \ref{fig:20_curve} shows the probabilities for a 20 land limited deck. Once you have evaluated how important it is for your deck to hit each land drop on curve, you can use the following tables to find the probabilities of making those land drops after drawing your opening hand. 

 	\begin{figure}[!ht]
	\centering
	\centerline{\includegraphics[width=1.\linewidth]{/home/skbarcus/Projects/MtG/MTGA_Opening_Hands/Pictures/Analytic_X_Lands_on_X_D40L14_Clean.png}}
	\captionof{table}{
	{\bf{Probabilities of Hitting Land Drops on Curve for a 14 Land Limited Deck.}} Probabilities are given for a 40 card limited deck with 14 lands. `\% 2 Lands on 2' means the percentage chance of having drawn two lands total on turn two. `2 Lands on the Draw' means that the opening seven card hand contains two lands and the player is on the draw.}
	\label{fig:14_curve}
	\end{figure}	
	
	\begin{figure}[!ht]
	\centering
	\centerline{\includegraphics[width=1.\linewidth]{/home/skbarcus/Projects/MtG/MTGA_Opening_Hands/Pictures/Analytic_X_Lands_on_X_D40L15_Clean.png}}
	\captionof{table}{
	{\bf{Probabilities of Hitting Land Drops on Curve for a 15 Land Limited Deck.}} Probabilities are given for a 40 card limited deck with 15 lands. `\% 2 Lands on 2' means the percentage chance of having drawn two lands total on turn two. `2 Lands on the Draw' means that the opening seven card hand contains two lands and the player is on the draw.}
	\label{fig:15_curve}
	\end{figure}	
	
	\begin{figure}[!ht]
	\centering
	\centerline{\includegraphics[width=1.\linewidth]{/home/skbarcus/Projects/MtG/MTGA_Opening_Hands/Pictures/Analytic_X_Lands_on_X_D40L16_Clean.png}}
	\captionof{table}{
	{\bf{Probabilities of Hitting Land Drops on Curve for a 16 Land Limited Deck.}} Probabilities are given for a 40 card limited deck with 16 lands. `\% 2 Lands on 2' means the percentage chance of having drawn two lands total on turn two. `2 Lands on the Draw' means that the opening seven card hand contains two lands and the player is on the draw.}
	\label{fig:16_curve}
	\end{figure}	
	
	\begin{figure}[!ht]
	\centering
	\centerline{\includegraphics[width=1.\linewidth]{/home/skbarcus/Projects/MtG/MTGA_Opening_Hands/Pictures/Analytic_X_Lands_on_X_D40L17_Clean.png}}
	\captionof{table}{
	{\bf{Probabilities of Hitting Land Drops on Curve for a 17 Land Limited Deck.}} Probabilities are given for a 40 card limited deck with 17 lands. `\% 2 Lands on 2' means the percentage chance of having drawn two lands total on turn two. `2 Lands on the Draw' means that the opening seven card hand contains two lands and the player is on the draw.}
	\label{fig:17_curve}
	\end{figure}	
	
	\begin{figure}[!ht]
	\centering
	\centerline{\includegraphics[width=1.\linewidth]{/home/skbarcus/Projects/MtG/MTGA_Opening_Hands/Pictures/Analytic_X_Lands_on_X_D40L18_Clean.png}}
	\captionof{table}{
	{\bf{Probabilities of Hitting Land Drops on Curve for a 18 Land Limited Deck.}} Probabilities are given for a 40 card limited deck with 18 lands. `\% 2 Lands on 2' means the percentage chance of having drawn two lands total on turn two. `2 Lands on the Draw' means that the opening seven card hand contains two lands and the player is on the draw.}
	\label{fig:18_curve}
	\end{figure}	
	
	\begin{figure}[!ht]
	\centering
	\centerline{\includegraphics[width=1.\linewidth]{/home/skbarcus/Projects/MtG/MTGA_Opening_Hands/Pictures/Analytic_X_Lands_on_X_D40L19_Clean.png}}
	\captionof{table}{
	{\bf{Probabilities of Hitting Land Drops on Curve for a 19 Land Limited Deck.}} Probabilities are given for a 40 card limited deck with 19 lands. `\% 2 Lands on 2' means the percentage chance of having drawn two lands total on turn two. `2 Lands on the Draw' means that the opening seven card hand contains two lands and the player is on the draw.}
	\label{fig:19_curve}
	\end{figure}	
	
	\begin{figure}[!ht]
	\centering
	\centerline{\includegraphics[width=1.\linewidth]{/home/skbarcus/Projects/MtG/MTGA_Opening_Hands/Pictures/Analytic_X_Lands_on_X_D40L20_Clean.png}}
	\captionof{table}{
	{\bf{Probabilities of Hitting Land Drops on Curve for a 20 Land Limited Deck.}} Probabilities are given for a 40 card limited deck with 20 lands. `\% 2 Lands on 2' means the percentage chance of having drawn two lands total on turn two. `2 Lands on the Draw' means that the opening seven card hand contains two lands and the player is on the draw.}
	\label{fig:20_curve}
	\end{figure}	
	
\section{Should I Run Fewer Lands in MTGA Bo1 Games?}

\bibliographystyle{plain}
\bibliography{mybibfile}

\end{document}